 \chapter{Introduction} \label{intro}
 \section{Background}
 Traditional models of economics, psychology, and other sciences assume rational models of agents. Individuals in the real world, however, rely on heuristics with biases which appear to result in temporally inconsistent choices, false beliefs, and seemingly suboptimal planning \citep{Kahneman1979}. Often, humans make choices that change over time, do not serve their self-interest, and conflict with their stated preferences. \\

 An important application of machine learning and probabilistic programming is to essentially understand the underlying cognitive processes that characterize and explain these seemingly `sub-optimal' choices \citep{Evans2015}. The need to understand the latent factors influencing these choices has spurred research in economics, psychology, and marketing, with substantial work done on modeling and predicting future human behavior based on past behavior and certain latent factors. As machine learning and AI continue to find applicability in automating personal tasks and in interacting with people, it becomes important that these models are able to understand the nuances in human choices to better predict future behavior. To this end, developing probabilistic models of future choices based on past behavior has become a topical and important problem.\\

 Research in economics and marketing has led to several computational models that attempt to model human behavior from their exhibited choices \citep{Szenberg2009}, however, these models haven't stemmed from psychological models of human choice. Concurrently, several psychological models of human choice exist, however, these haven't been applied to the problem of behavioral model inference from observed choices \citep{Busemeyer2005, Train2003, shenoy2013rational}. Alternatively, computational models in Inverse Reinforcement Learning \citep{ng2000algorithms} and Bayesian Inverse Planning \citep{baker2009action} attempt to infer preferences by inverting a rational decision-making model based on a utility function \citep{russell1995modern}. These models, however, haven't been able to capture the full spectrum of the seeming irrationalities underlying human decision models. Their shortcomings, traditionally, have stemmed from ignoring the latent factors spurring the bounded rationality, usually by modeling these inconsistencies as noise \citep{Kim2014, Zheng2014}.\\ 

 Recent developments \citep{Evans2015, Jern2017} have attempted to correct these simplifications by explicitly accounting for the biases and heuristics developed in psychological models of human choice \citep{Kahneman1979}. Progress on this frontier has included structural modifications to the utility functions that incorporate the biases and heuristics. For example, hyperbolic discounting \citep{ainslie2001breakdown} is used to capture the tendency for temporally inconsistent choices  while uncertainty and false beliefs affecting choices are modeled through probability distributions over knowledge of the state of the world, instead of assuming agents fully know the current state of the world \citep{baker2014modeling}.\\

 Despite the success in explaining inconsistencies in choices as compared to human explanations, there remains substantial work to be done on defining expressive and representative computational frameworks. These frameworks need to be expressive and reliable enough to help AI, and human researchers, better understand human behavior. 

 \section{Motivation}
 With the proliferation of connected devices like smartphones and wearable technology, in tandem with the continued expansion and domination of the web by the attention economy, the digital world has become ever more addictive. When put to the question, experts' prediction on the future generations comes in at a gloomy conclusion: with the exception of the few capable of retaining attention and focus, many participants in the digital economy will succumb to the continuous distractions awarded by addictive, rewarding applications \citep{anderson2012millennials}. While the benefits of using the internet to support and inform our daily decisions in a hyperconnected world where access to, and verification of, information is unprecedented are largely positive, the experts surveyed in the Pew report predict that teenagers will face difficulties originating from the growing dependence and thirst for instant gratification awarded by social media and made profitable by the attention economy.\\

 The report further surveys these experts about what they consider the most desirable skills for people in 2020, and the ability to concentrate, amongst others, was emphasized. Tiffany Shalin, founder of the Webby Awards and director of \textit{Connected} remarks that a large divide lies between participants in the digital economy that develop the ability to properly allocate attention in this new environment and those that do not, with the reward being successfully navigating and participating in this hyperconnected environment.\\

 Coupled with the rapid advance of machine intelligence and personal assistants that take on more attention-intensive tasks, argues John Smart in the report, the productivity and desire of the young generation to work will be affected, with the result being the increasing willingness to be distracted and consumed by the desire for instant gratification.  With human productivity at risk, it comes as no surprise that a market has emerged for attention-aware assistants. Many applications promising to help maintain and improve attention have flooded the market in response. Some, like RescueTime, offer the ability to track usage statistics, while others take more hard-line approaches like blocking distracting applications and websites and/or offering rewards and punishments to users \citep{Lyngs}.

 The potential implications of attention-aware computational models are far-reaching. For example, models capable of predicting whether a person will switch from the required task at hand could find usage in predicting when drivers will become distracted by their devices or when a student writing his thesis will switch to binge watching movies. Models with such capabilities can then alert the driver or take action, potentially avoiding life-threatening accidents. Moreover, in an attention economy \citep{davenport2013attention} where many users report reduced productivity due to constant distractions, such models could be used in anti-distraction applications that aim to improve productivity by ensuring people remain focused on their primary tasks. With effects ranging from fatal catastrophes to reduced productivity, applications that help maximize attention have a role to play in the digital economy. Among many other functions, these applications should be able to understand the latent cognitive state people are in by observing their app-usage behaviour, and make predictions on what their future activities will look like. To better help people focus and maintain attention, it is important to be able to understand the behaviour patterns of users and differentiate between temporary interruptions and distractions \citep{Lyngs}. For example, advice on future action given to a user when they utilize an Instant Messaging application should be different when the user will likely quickly respond to the message and return to their primary task versus when their current behaviour, based on past experiences, suggest they are likely to become fully distracted and abandon the task they should be focusing on. 

 To this end, this thesis project will attempt to develop a set of models that are able to achieve two things: first, understand and generate the latent cognitive space within which people operate. By capturing these states and understanding how users use their applications we will be better able to understand the extract the heuristics that guide a users behaviour and attention models. Second, these models should be able to predict what future applications a user will likely resort to. By developing models that offer these predictive capabilities, attention management applications will be better able to guide users in increasing their productivity. The occasional break to use social media may not be bad, and may even be beneficial if it means the user will resume working; potentially more effectively due to the break. However, getting consumed by the different apps vying for a users attention, which may ultimately lead the user to abandoning the task at hand is negative, and attention-management applications would benefit tremendously from the ability to predict this. Moreover, by analyzing the predictions of these computational models and their accuracy, we can further learn more about the characteristics and properties of the latent factors affecting task-switching behavior, allowing us to improve our existing knowledge on human attention. These high-impact ramifications serve to motivate the design and analysis of such predictive models.\\

 \section{Proposed Approach}

 Several approaches exist to model attention, such as models of optimal information gain \citep{pirolli1999information}, drive and multi-tasking where people are working on several tasks and jointly trying to address them all. These approaches attempt to explain why and when we switch tasks by modelling attention directly. However, for my project, I intend to frame attention management as a problem of sequential decision making. Under this assumption, human attention involves `choices' on which current task at hand to focus on. Distractions, temptations and temporal inconsistencies influence agents when choosing which tasks to focus on. When viewed as such, the process of switching tasks can then be framed as one of sequential decision making where attention models are subject to decision and choice theory. As such, these `choices' are made based on heuristics and influenced by our internal cognitive states. The `choice' of whether to continue focusing on the current task at hand or to switch to another task is subject to several latent factors. For example, switching from the current task of work to leisure (e.g. social media) could mean the agent is done with their work, but could alternatively be due to distractions or fatigue.\\

 In approaching the task of modelling attention this thesis will focus on two distinct tasks:

 \begin{itemize}
     \item Uncovering the latent generative space of a user's app-usage behaviour
     \item Predicting how a user will spend their time across their different applications.
 \end{itemize}

 To model the latent generative space, and make predictions about future app-usage, the models will abstract away from attempting to manually devise and encode the set of heuristics agents use in deciding which app to use next. Instead, it is assumed that the heuristics employed can be captured through the patterns displayed in app-usage behaviour. To illustrate let us consider the example of a day in the life of a student. Consider a person working on her thesis. Upon waking she may check her social media feeds before meditating and having breakfast. In choosing which social media apps to use, our student may employ some heuristics. An example would be check channels that allow her to get a quick summary of global news, and consequently she uses Twitter. For meditating, our student may use the app Headspace, and choose to have one or two sessions depending on what her previous experience has led her to believe is most optimal regarding benefit vs time-spent not working on her thesis. Next, she may work until noon, where she has lunch. It is possible that our example student enjoys watching a few shows on Netflix while she eats. The number of shows she watches depends on the type of show and a myriad of heuristics she's developed over time that fit within the current context. Her day proceeds similarly where at each point in time she makes choices on what apps to use based on rules-of-thumb she's subconsciously developed over time. Ultimately, the underlying heuristics express themselves through the pattern of app-usage behaviour displayed over time. \\

 Essentially, this thesis makes the assumption that models of human task-switching and attention, along with the heuristics that guide them, can be developed using only previous observations on said individuals app-usage behaviour. To test this hypothesis, this thesis builds out a set of models that are trained on data collected by logging a user's app-usage across their devices. The models will be trained with the objective of predicting future app-usage behaviour given the data and uncovering the discussed latent space. These models will be trained on a real-life dataset of tasks attended to by people obtained from RescueTime, an application that tracks a users active applications on their devices. The logged data by RescueTime will be collected and processed into time-series data depicting the tasks the participant attended to. The input to the models will be only the labeled time-series data of tasks the agent is focusing on. As such, low-level factors such as lighting, location, mental/physical well-being, etc of the agents will be ignored. The task will strictly be to predict task-switching based on information logged by RescueTime (e.g. time of day, time elapsed since activity started, hours spent working during that day, etc). The class of predictive models will be assessed via quantitatively measuring their accuracy in predicting future app-usage behaviour. The effectiveness of the generative models in uncovering the latent space will be visualised and assessed qualitatively.

 In essence, this project will attempt to explore three questions:

 \begin{itemize}
     \item Can human attention be modeled as a sequential decision making problem and consequently captured by sequential models?
     \item Can the underlying latent states influencing the observed behaviour be learned?
     \item Does accounting for these underlying latent states help improve predictive capabilities?
 \end{itemize}


 \section{Structure of Thesis}
 % The rest of the thesis is structured as follows: First, in Chapter \ref{data} we present RescueTime, the app-activity logger and discuss it's features and limitations. An overview of how the data is collected and processed is also presented. In Chapter \ref{predmod} a class of sequential models capable of capturing the temporal dependencies in the data will be explored. The background of these models will be introduced, their contemporary applications and where they've been successful will also be briefly covered. A detailed description of the theory of these models will be presented followed by a justification of why they are chosen for this modeling task. Next, in Chapter \ref{genmod} a class of generative models that are capable of inferring generative latent spaces will be similarly explored, with a discussion on their theoretical underpinning, contemporary success and applicability to the defined problem similarly examined. Following these two chapters, in Chapter \ref{genpredmod} we discuss the recent attempts made in merging these two classes of models, and present an implementation for combining them for our purposes. in Chapter \ref{res} we present the findings of this thesis and discuss their implications and validity. Chapter \ref{conc} concludes this paper, summarising the contributions and findings of this thesis and exploring future directions of research.